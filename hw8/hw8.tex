%! compile with xelatex
\documentclass[11pt, a4paper]{article}

% fonts: Make sure you have SimSun and SimHei installed on your system
\usepackage{xeCJK}
\setCJKmainfont[BoldFont=SimHei]{SimSun}
\setCJKfamilyfont{hei}{SimHei}
\setCJKfamilyfont{kai}{KaiTi}
\newcommand{\hei}{\CJKfamily{hei}}
\newcommand{\kai}{\CJKfamily{kai}}

% style
\usepackage[top=2.54cm, bottom=2.54cm, left=3.18cm, right=3.18cm]{geometry}
\usepackage{indentfirst}
\linespread{1.5}
\parindent 2em
\punctstyle{quanjiao}
\renewcommand{\today}{\number\year 年 \number\month 月 \number\day 日}

\usepackage{amsmath}
\usepackage{amssymb}
\usepackage{xcolor}
\usepackage{hyperref}

\newtheorem{lemma}{引理}
\usepackage[inference]{semantic} % for writing inference rules
\setnamespace{0pt}

\usepackage{enumitem}
\setlist{nolistsep}

\usepackage{empheq}
\newcommand{\boxedeq}[1]{\begin{empheq}[box=\fbox]{align*}#1\end{empheq}}

\newtheorem{Proof}{Proof}
\newtheorem{Solution}{Solution}

% A bunch of handy commands
\let\t\texttt
\let\emptyset\varnothing
\let\to\rightarrow
\let\reduce\Rightarrow
\newcommand{\reduceM}{\Rightarrow^{*}}
\let\defas\triangleq
\newcommand{\Bool}{\t{Bool}}
\newcommand{\kword}[1]{{\color{blue} \textsf{#1}}}
\newcommand{\True}{\kword{true}}
\newcommand{\False}{\kword{false}}
\newcommand{\If}{\kword{if}}
\newcommand{\Then}{\kword{then}}
\newcommand{\Else}{\kword{else}}

\title{\textbf{Haskell 第八次作业}}
\author{软件62\quad  2016013258\quad  王泽宇\\ Email:ycdfwzy@outlook.com}
\date{\today}

\begin{document}

\maketitle

% {\hei 注意:本次作业均为非编程题,请直接在网络学堂提交一份 PDF。前两题为必做题,这两题全部做对即得满分100分。
% 后两题为选做题,每题25分,视回答情况酌情加分。}

\section{$\lambda_\to$ 扩展}

    \subsection*{1.1}
    \begin{Solution}
        共需要$2k+3$步,推导过程如下:
        \begin{align}
            &\quad \If~t_0~\Then~((\lambda x:\Bool.x)~t_0)~\Else~((\lambda x:\Bool.\If~x~\Then~\False~\Else~\True)~t_0) \notag \\
            &\Rightarrow \If~\False~\Then~((\lambda x:\Bool.x)~t_0)~\Else~((\lambda x:\Bool.\If~x~\Then~\False~\Else~\True)~t_0) \tag{$by~t_0\Rightarrow_k~\False$} \\
            &\Rightarrow (\lambda x:\Bool.\If~x~\Then~\False~\Else~\True)~t_0 \tag{by~E-IfFalse} \\
            &\Rightarrow \If~t_0~\Then~\False~\Else~\True \tag{$by~\beta-reduction$} \\
            &\Rightarrow \If~\False~\Then~\False~\Else~\True \tag{$by~t_0\Rightarrow_k~\False$} \\
            &\Rightarrow \True \tag{by~E-IfFalse}
        \end{align}
        一共用了$k+1+1+k+1=2k+3$步。
    \end{Solution}
    \subsection*{1.2}
    \begin{Proof}
        \begin{align*}
            \inference[T-If]{\inference[T-var]{x:\Bool \in \Gamma_1}{ x:\Bool \vdash x:\Bool} & \inference[T-False]{}{\emptyset \vdash \False : \Bool}}{\emptyset,x:\Bool \vdash (\If~x~\Then~\False~\Else~x):\Bool} \tag{M}
        \end{align*}
        其中$\Gamma_1 = x:\Bool$
        \begin{align*}
            \inference[T-var]{f:\Bool\rightarrow \Bool \in \Gamma_2}{f:\Bool\rightarrow \Bool\vdash f : \Bool\rightarrow \Bool} \tag{N}
        \end{align*}
        其中$\Gamma_2 = f:\Bool\rightarrow \Bool$
        \begin{align*}
            \inference[T-Abs]{\inference[T-App]{(N) ~ (M) }{\emptyset,x:\Bool,f:\Bool\rightarrow \Bool \vdash f~(\If~x~\Then~\False~\Else~x) : \Bool}}{\emptyset, f:\Bool \to \Bool \vdash \lambda x:\Bool.f~(\If~x~\Then~\False~\Else~x) : \Bool \to \Bool}
        \end{align*}
    \end{Proof}
    \subsection*{1.3}
    \begin{Solution}
        推导过程如下啊
        \begin{itemize}
            \item 设$\Gamma=\emptyset, f:T_f, x:T_x, y:T_y$,那么有$$T_f\rightarrow T_x\rightarrow T_y = \Bool$$
            \item $y$应用在$f~x$上,所以$T_f\rightarrow T_x = T_y\rightarrow A$,其中$A=\Bool$
            \item $x$应用在$f$上,所以$T_f = T_x\rightarrow B$,其中$B=T_y\rightarrow A$
            \item 接上面的方程组,得$T_f=T_x\rightarrow T_y\rightarrow \Bool$,所以$$\Gamma =\emptyset,~x:T_x,~y:T_y,f: T_x\rightarrow T_y\rightarrow \Bool$$
        \end{itemize}
    \end{Solution}

\section{类型推断}
    \subsection*{2.1}
    \begin{Solution} 从Haskell中的定义得到
        \begin{align*}
            (.)~&::~(b\rightarrow c)\rightarrow (a\rightarrow b)\rightarrow a\rightarrow c \\
            (\$)~&::~(a\rightarrow b)\rightarrow a \rightarrow b
        \end{align*}
        对于$(\$)_1.(\$)_2$设
        \begin{align*}
            \Gamma_1 &= (.)~::~(b_0\rightarrow c_0)\rightarrow (a_0\rightarrow b_0)\rightarrow a_0\rightarrow c_0 \\
            \Gamma_2 &= (\$)_1~::~(a_1\rightarrow b_1)\rightarrow a_1 \rightarrow b_1 \\
            \Gamma_3 &= (\$)_2~::~(a_2\rightarrow b_2)\rightarrow a_2 \rightarrow b_2
        \end{align*}
        推导过程如下:
        \begin{align}
            &\inference[CT-Var]{(.)~::~(b_0\rightarrow c_0)\rightarrow (a_0\rightarrow b_0)\rightarrow a_0\rightarrow c_0 \in \Gamma_1}{\Gamma_1 \vdash (.)~::~(b_0\rightarrow c_0)\rightarrow (a_0\rightarrow b_0)\rightarrow a_0\rightarrow c_0 \blacktriangleleft \emptyset} \tag{M.1} \\
            &\inference[CT-Var]{(\$)_1~::~(a_1\rightarrow b_1)\rightarrow a_1 \rightarrow b_1 \in \Gamma_2}{\Gamma_2 \vdash (\$)_1~::~(a_1\rightarrow b_1)\rightarrow a_1 \rightarrow b_1 \blacktriangleleft \emptyset} \tag{M.2} \\
            &\inference[CT-Var]{(\$)_2~::~(a_2\rightarrow b_2)\rightarrow a_2 \rightarrow b_2 \in \Gamma_3}{\Gamma_3 \vdash (\$)_2~::~(a_2\rightarrow b_2)\rightarrow a_2 \rightarrow b_2 \blacktriangleleft \emptyset} \tag{M.3} \\
            &\inference[CT-App]{\inference[CT-App]{(M.1)~(M.2)}{\Gamma_1, \Gamma_2 \vdash (.)(\$)_1 :: (a_0\rightarrow b_0)\rightarrow a_0 \rightarrow c_0 \blacktriangleleft \Phi_1} & (M.3)}{\Gamma_1, \Gamma_2, \Gamma_3 \vdash (.) (\$)_1 (\$)_2 :: (a_2\rightarrow b_2)\rightarrow a_2 \rightarrow b_2 \blacktriangleleft \Phi} \notag
        \end{align}
        其中$\Phi_1=\{b_0=a_1\rightarrow b_1,~c_0=a_1\rightarrow b_1\},\Phi=\Phi_1 \cup \{a_0=a_2\rightarrow b_2,~b_0=a_2\rightarrow b_2\}$。如果把$\rightarrow$替换为逻辑蕴含,那么$(a_2\rightarrow b_2)\rightarrow a_2 \rightarrow b_2$仍然是有效命题。
    \end{Solution}
    \subsection*{2.2}
    \begin{Solution}
        对于$(.)_1\$(.)_2$设
        \begin{align*}
            \Gamma_1 &= (\$)~::~(a_0\rightarrow b_0)\rightarrow a_0 \rightarrow b_0 \\
            \Gamma_2 &= (.)_1~::~(b_1\rightarrow c_1)\rightarrow (a_1\rightarrow b_1)\rightarrow a_1\rightarrow c_1 \\
            \Gamma_3 &= (.)_2~::~(b_2\rightarrow c_2)\rightarrow (a_2\rightarrow b_2)\rightarrow a_2\rightarrow c_2
        \end{align*}
        推导过程如下:
        \begin{align}
            &\inference[CT-Var]{(\$)~::~(a_0\rightarrow b_0)\rightarrow a_0 \rightarrow b_0 \in \Gamma_1}{\Gamma_1 \vdash (\$)~::~(a_0\rightarrow b_0)\rightarrow a_0 \rightarrow b_0 \blacktriangleleft \emptyset} \tag{M.1} \\
            &\inference[CT-Var]{(.)_1~::~(b_1\rightarrow c_1)\rightarrow (a_1\rightarrow b_1)\rightarrow a_1\rightarrow c_1 \in \Gamma_2}{\Gamma_2 \vdash (.)_1~::~(b_1\rightarrow c_1)\rightarrow (a_1\rightarrow b_1)\rightarrow a_1\rightarrow c_1 \blacktriangleleft \emptyset} \tag{M.2} \\
            &\inference[CT-Var]{(.)_2~::~(b_2\rightarrow c_2)\rightarrow (a_2\rightarrow b_2)\rightarrow a_2\rightarrow c_2 \in \Gamma_3}{\Gamma_3 \vdash (.)_2~::~(b_2\rightarrow c_2)\rightarrow (a_2\rightarrow b_2)\rightarrow a_2\rightarrow c_2 \blacktriangleleft \emptyset} \tag{M.3} \\
            &\inference[CT-App]{\inference[CT-App]{(M.1)(M.2)}{\Gamma_1, \Gamma_2 \vdash (\$)(.)_1~::~T_1 \blacktriangleleft \Phi_1} & (M.3)}{\Gamma_1, \Gamma_2, \Gamma_3 \vdash (\$) (.)_1 (.)_2~::~T_2  \blacktriangleleft \Phi} \notag
        \end{align}
        其中,$T_1=a_0\rightarrow b_0, T_2=(a_1\rightarrow b_2 \rightarrow c_2)\rightarrow a_1 \rightarrow (a_2\rightarrow b_2)\rightarrow a_2 \rightarrow c_2$,$\Phi_1=\{a_0=b_1\rightarrow c_1,~b_0=(a_1\rightarrow b_1)\rightarrow a_1\rightarrow c_1\}$,$\Phi=\Phi_1\cup \{a_0=(b_2\rightarrow c_2)\rightarrow (a_2\rightarrow b_2)\rightarrow a_2\rightarrow c_2\}$。如果把$T_2$中的$\rightarrow$替换为逻辑蕴含,那么$T_2$仍然是有效命题。
    \end{Solution}
    \subsection*{2.3}
    \begin{Solution}
        从Haskell中的定义得到
        \begin{align*}
            map~&::~(a\rightarrow b)\rightarrow [a]\rightarrow [b] \\
            tail~&::~[a]\rightarrow [a]
        \end{align*}
        对于\t{\textbackslash c f -> if c then map f else tail},设
        \begin{align*}
            \Gamma_1 &= map~::~(a\rightarrow b)\rightarrow [a]\rightarrow [b] \\
            \Gamma_2 &= tail~::~[d]\rightarrow [d] \\
            \Gamma_3 &= c~::~T_c \\
            \Gamma_4 &= c~::~T_f
        \end{align*}
        推导过程如下
        \begin{align}
            &\inference[CT-Var]{map~::~(a\rightarrow b)\rightarrow [a] \rightarrow [b] \in \Gamma_1}{\Gamma_1 \vdash map~::~(a\rightarrow b)\rightarrow [a]\rightarrow [b] \blacktriangleleft \emptyset} \tag{M.1} \\
            &\inference[CT-Var]{tail~::~[c]\rightarrow [c] \in \Gamma_2}{\Gamma_2 \vdash tail~::~[c]\rightarrow [c] \blacktriangleleft \emptyset} \tag{M.2} \\
            &\inference[CT-Var]{c~::~T_c \in \Gamma_3}{\Gamma_3 \vdash c~::~T_c \blacktriangleleft \emptyset} \tag{M.3} \\
            &\inference[CT-Var]{f~::~T_f \in \Gamma_4}{\Gamma_4 \vdash f~::~T_f \blacktriangleleft \emptyset} \tag{M.4} \\
            &\inference[CT-Abs]{\inference[CT-ITE]{\inference[CT-App]{(M.1)(M.4)}{\Gamma_1, \Gamma_4 \vdash map~f~::~[a]\rightarrow [b] \blacktriangleleft \Phi_2} & (M.2)(M.3)}{\Gamma_1, \Gamma_2, \Gamma_3, \Gamma_4 \vdash \If~c~\Then~map~f~\Else~tail~::~[a] \blacktriangleleft \Phi_1}}{\Gamma_1, \Gamma_2 \vdash \textbackslash c~f\rightarrow \If~c~\Then~map~f~\Else~tail~::~[c]\rightarrow [c] \blacktriangleleft \Phi_1} \notag
        \end{align}
        其中,$\Phi_2=\{T_f=a\rightarrow b\},\Phi_1=\Phi_2\cup \{T_c=\Bool,~[a]\rightarrow [b]=[c] \rightarrow [c]\}$
    \end{Solution}

\section{扩展系统的元性质判定(选做题)}
    \subsection*{3.1}
    \begin{Solution}
        引理1不正确,考虑$(\lambda x:\Bool.x)~\False$,第一步应用E-App1规约,得到$\t{foo}~\False$,第二应用E-Foo2,得到$\True~\False$。如果第一步应用E-$\beta$规约,得到$\False$。显然两种规约方式得到的结果不同,规约具有不确定性。\\
        引理2正确。\\
        引理3不正确,反例同样取$(\lambda x:\Bool.x)~\False$。
    \end{Solution}
    \subsection*{3.2}
    \begin{Solution}
        引理1正确,\\
        引理2不正确。考虑$((\lambda x.x)(\lambda x.x))\True$,由于没E-App1,所以无法进行规约;这也不是一个value。\\
        引理3正确。
    \end{Solution}
    \subsection*{3.3}
    \begin{Solution}
        引理1正确,\\
        引理2正确。\\
        引理3不正确。考虑$(\lambda x:Bool.(\lambda y:Bool.y))~\False$,按照T-FunnyApp,这是一个$\Bool$类型的表达式,如果进行一步规约,得到$\lambda y:Bool.y$,是一个$\Bool\rightarrow \Bool$类型的表达式,类型没有得到保持。
    \end{Solution}
    \subsection*{3.4}
    \begin{Solution}
        引理1正确,\\
        引理2正确。\\
        引理3正确。考虑$t=(\lambda x:\Bool.(\lambda y:\Bool.\True))\False$,根据原来的规则,容易得到$t:\Bool \rightarrow \Bool$。在用$\beta$规约,可以得到$t\Rightarrow \lambda y:\Bool.\True = t_1$,根据新规则T-FunnyAbs,$t_1:Bool$,$t$和$t_1$的类型不同,所以违反了引理3。
    \end{Solution}
    \subsection*{3.5}
    \begin{Solution}
        引理1正确,\\
        引理2不正确。考虑$\True~\False$,根据T-FunnyApp'规则,他是一个$\Bool$类型的,但是这不是一个value,也不可以进行规约。\\
        引理3正确。
    \end{Solution}

\section{System F 的元性质证明(选做题)}
    \subsection*{引理4~F-progress}
    \begin{Proof}
        下面证明在F中新增的Term满足引理4。
        \begin{itemize}
            \item $s=\lambda X.t$。由于F系统中定义了$\lambda X.t$是一个value,所以$s=\lambda X.t$满足引理4。
            \item $s=t[T]$。如果$t$不是一个value,那么$t\Rightarrow t'$,根据E-TApp,$s=t[T]\Rightarrow t'[T]$。如果$t$是value,即$t=\lambda X.t$,根据E-T$\beta$规则,$s=t[T]=(\lambda X.t)[T]\Rightarrow t[X:=T]$。
        \end{itemize}
        综上,F系统满足引理4。
    \end{Proof}
    \subsection*{引理5~F-preservation}
    \begin{Proof}
        下面证明在F中新增的Term满足引理5。
        \begin{itemize}
            \item $s=\lambda X.t$,设$t:T$,那么$s:\forall X.T$。由于$s$不可规约,所以不违反引理5。
            \item $s=t[T]$,按$t$的不同,分为以下两种情况:
            \begin{itemize}
                \item $t=\lambda X.t_1$,设其中$t_1:T_1$,由T-TAbs知$t=\lambda X.t_1:\forall X.T_1$。根据E-T$\beta$规约,$s=(\lambda X.t_1)[T]\Rightarrow t_1[X:=T]$,再由T-TApp类型规则知$t_1[X:=T]:T_1[X:=T]$。
                \item $t=t_1[T_1]$,即$t$可以继续进行规约,设$t\Rightarrow t'$。用归纳法可知$t,t'$具有相同的类型,假设为$\forall X.T'$,所以$s$的类型为$T'[X:=T]$(T-TApp)。现在应用E-TApp,$s=t[T]\Rightarrow t'[T]$,同样由T-TApp得$t'[T]:T'[X:=T]$。所以$s=t[T]$与$t'[T]$具有相同的类型。
            \end{itemize}
        \end{itemize}
        综上,F系统满足引理5。
    \end{Proof}

% {\kai 提示:本题工作量较大,但只要按部就班地开展证明也不难。根据证明的完整性评分。}

% 与 $\lambda_\to$ 类似,$F$ 也具有 progress 和 preservation 的元性质:

% \begin{lemma}
%     ($F$-progress) For any term $t$ and type $T$, if $\vdash t : T$,
%     then either $t$ is a value, or $t \reduce t'$ for some term $t'$.
% \end{lemma}

% \begin{lemma}
%     ($F$-preservation) If $\vdash t:T$ and $t \reduce t'$, then $\vdash t':T$.
% \end{lemma}

% 请先阅读 \textit{Types and Programming Languages} 定理 9.3.5 ($\lambda_\to$-progress) 以及引理 9.3.4 的证明,还有定理 9.3.9 ($\lambda_\to$-preservation) 的证明,仿照该方法证明 $F$ 满足上述两个元性质。尽你所能给出所有必要的证明细节。注:你可以采用不同于讲义中的符号系统,但请在证明之前给出所有要用到的定义和规则。

\end{document}
