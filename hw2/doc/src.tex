%!TEX program = xelatex
\documentclass[UTF8]{article}
\usepackage{ctex}
\usepackage{xfrac}
\usepackage{amsmath}
\usepackage{amssymb}
\usepackage{listings}
\usepackage{xcolor}
\usepackage{graphicx}
\usepackage{fontspec}
\usepackage[cache=false]{minted}

\newtheorem{theorem}{Theorem}
\newtheorem{lemma}{Lemma}
\newtheorem{Proof}{Proof}
\newtheorem{Solution}{Solution}

\title{HASKELL \\ \Large{第二次课后作业}}
\author{软件62王泽宇 \qquad 学号:2016013258 \\ Email:ycdfwzy@outlook.com}
\date{\today}
\begin{document}
\maketitle
\section{第一题}
    \subsection*{1.}
        \begin{Solution}
            \begin{equation*}
                \begin{split}
                    k u v &\Rightarrow (\lambda x.\lambda y.x)uv\\
                    &\Rightarrow (\lambda y.u)v\\
                    &\Rightarrow u
                \end{split}
            \end{equation*}
        \end{Solution}
    \subsection*{2.}
        \begin{Solution}
            \begin{equation*}
                \begin{split}
                    s u v m &\Rightarrow (\lambda x.\lambda y.\lambda z.xz(yz))uvm\\
                    &\Rightarrow (\lambda y.\lambda z.uz(yz))vm\\
                    &\Rightarrow (\lambda z.uz(vz))m\\
                    &\Rightarrow um(vm)
                \end{split}
            \end{equation*}
        \end{Solution}
    \subsection*{3.}
        \begin{Solution}
            \begin{equation*}
                \begin{split}
                    s k k m &\Rightarrow (\lambda x.\lambda y.\lambda z.xz(yz))kkm\\
                    &\Rightarrow (\lambda y.\lambda z.kz(yz))km\\
                    &\Rightarrow (\lambda z.kz(kz))m\\
                    &\Rightarrow km(km)\\
                    &\Rightarrow (\lambda x.\lambda y.x)m(km)\\
                    &\Rightarrow (\lambda y.m)(km)\\
                    &\Rightarrow m
                \end{split}
            \end{equation*}
        \end{Solution}
\section{第二题}
    \subsection*{1.}
        \begin{Solution}
            归约过程如下,可以看到,如果一直做$\beta$-reduction下去,lambda项会一直是$(\lambda x.x x)(\lambda x.x x)$,归约过程无法终止。
            \begin{equation*}
                \begin{split}
                    \omega \omega &\Rightarrow (\lambda x.x x)(\lambda x.x x) \\
                    &\Rightarrow (\lambda x.x x)[x:=\lambda x.x x]\\
                    &\Rightarrow (\lambda x.x x)(\lambda x.x x)\\
                    &\vdots
                \end{split}
            \end{equation*}
        \end{Solution}
    \subsection*{2.}
        \begin{Solution}
            归约过程如下,可以看到,如果一直做$\beta$-reduction下去,lambda项的长度会越来越大,归约过程无法终止。
            \begin{equation*}
                \begin{split}
                    d d &\Rightarrow (\lambda x.x x x)(\lambda x.x x x) \\
                    &\Rightarrow (\lambda x.x x x)[x:=\lambda x.x x x]\\
                    &\Rightarrow (\lambda x.x x x)(\lambda x.x x x)(\lambda x.x x x)\\
                    &\Rightarrow (\lambda x.x x x)(\lambda x.x x x)[x:=\lambda x.x x x]\\
                    &\Rightarrow (\lambda x.x x x)(\lambda x.x x x)(\lambda x.x x x)(\lambda x.x x x)\\
                    &\vdots
                \end{split}
            \end{equation*}
        \end{Solution}
    \subsection*{3.}
        \begin{Solution}
            归约过程如下。虽然$\omega \omega$在一种证明是会陷入无限归约的死循环中,但是$(\lambda x.\lambda y.y)(\omega \omega)m$按照从左向右结合归约,与$\omega \omega$的归约结果无关,所以可以在有限步数内归约结束。
            \begin{equation*}
                \begin{split}
                    (\lambda x.\lambda y.y)(\omega \omega)m &\Rightarrow (\lambda x.\lambda y.y)[x:=\omega\omega]m \\
                    &\Rightarrow (\lambda y.y)m\\
                    &\Rightarrow m
                \end{split}
            \end{equation*}
        \end{Solution}
\end{document}